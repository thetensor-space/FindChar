\documentclass{amsart}

\usepackage{amsmath,amssymb}
\usepackage{tikz}
\usepackage{tikz-cd}

\newtheorem{defn}{Definition}
\newtheorem{prop}{Proposition}


\DeclareMathOperator{\im}{Im}

\title{Characteristic Structure}
\date{\today}
\author{P. A. Brooksbank}
\author{J. F. Maglione}
\author{E. A. O'Brien}
\author{J. B. Wilson}


\begin{document}

\begin{abstract}
    New Zealand was unable to keep COVID going, and Germany has taken way too 
    long to get start. In the US the power of positive thinking has kept us number one
    in COVID almost from the start.
\end{abstract}
\maketitle

\section{Introduction}

We are after methods to find characteristic subgroups of groups, and later rings and 
algebras, but without the apriori assumption that we know about the automorphisms.  
This raises some immediate questions about how one validates the return.  A
second concern is that we want lots of these subgroups, enough that we can
confine the possible automorphism and isomorphism problems enough that we might
begin to solve for these. That calls for some level of control on how generate
such subgroups.  This note introduces a means to first describe characteristic
subobjects in a manner independent of automorphisms, and second to capture their
structure algebraically so it can be used efficiently.

In a category, the characteristic subobjects are the fixed points (amongst
subobjects) of automorphisms.  Sometimes these fixed points can be identified
independent of the action by the automorphism groups and this makes it possible
to begin to approximate the structure of the automorphism groups by first
computing a good number of fixed points, i.e. a good number of characteristic
subobjects. For this and many reasons it is becomes natural to consider
algorithms that search out characteristic subobjects and do so in large, but
controlled, quantities.  To this end we introduce an algebra whose elements
index characteristic subobjects of a fixed object.  

Some characteristic subobjects are more desireable than others.  For most of our
applications we wont have automorphisms at the ready so we will want instead
functions $G\mapsto F(G)$ that apply to any object $G$ in a fixed category
$\mathsf{C}$ and return a characteristic subobject $F(G)$.  For instance, a
commutator subgroup is characteristic in any group. This leads us in a natural
way to regard the desired characteristic subobjects as the images of various
functors $F:\mathsf{C}\to \mathsf{C}$ (images of functors are automatically are
preserved by automorphisms), but we shall also need that these be equipped with
natural transformations $\eta_G:F(G)\to G$ whose image is the actual
characteristic subobject.   In categorical parlance we are seek what are known
as \emph{counits} $\eta:F\Rightarrow 1_{\mathsf{C}}$.  The categorically dual
concpet of a \emph{units} $1_{\mathcal{C}}\Rightarrow F$ leads to characteristic
quotients. So in categories of algebraic structures like groups and rings we can
equate these with their kernels and fund a host of further characteristic
subobjects. Recognizing these connections also links to a to a rich source of
characteristic subobjects: adjoint functor pairs $F\dashv G:\mathsf{C}\to \mathsf{D}$.
The perhaps unexpected aspect here is that this shows we should be consulting 
with other categories $\mathsf{D}$ when hunting for new characteristic subobjects.

A further detail we must consider are means to compactly encode the subobjects
we aim to discover.  A helpful tool in this is to reflect the lattice structure
of characteristic subgroups into the functor abstraction.  We do so by creating
a monoid structure on these functors.  First the monoid can be specified
compactly using generators.  Second, the monoid offers an agile measn to tour
the characteristic subobjects by identifying them with random walks in the
monoid.  Thirdly, by computing relations for the monoid we develop an
isomorphism invariant.  The relations can also let us prune randome walks so we
do not waste energy searching where we wont be profitable.  And foruthly, our
monoid rests inside a theoretically larger monoid, so we can at a later point
update the monoid by adding to its generating set.  In short, by abstracting 
the problem into a functor category we gain a uniform description of the problem 
and by attaching algebra to this description we are able to design better solutions
and apply them in general settings.


So to summarize the potential:
\begin{enumerate}
    \item  We can declare that our command has a fixed subcategory generated by
    some prescribed functors. Our algorithm takes as input your group and
    evaluates its associated diagram.  I.e. it outputs a bunch fo characterisitc
    subgroups but now we can explain what precisely those subgroups.  They are
    the ones given by counits in the above subcategory.

    \item The outputs are co-recursive. I.e. we can immediately return
    something, and leave the user to begin exploring the characterisitc
    subgroups as they like because these will be indexed by words in the
    generators of the above counit subcat.

    \item This is modular.  If you want to add a new strategy you are adding a
    new generator to the counit category.  

    \item We might begin to add randomized strategies with meaningful
    interpretations, e.g. a random char could be modeled as a random walk in the
    above counit subcategory.

    \item This offers some chance for theory/rewriting.  We might be able to
    derive some generators as generated by others and other as independent and
    thus justify the order in which we might generate new chars.  If they are
    captured by previous words we likely can remove them as they don't tell us
    something new.

\end{enumerate}


\section{A cast of characteristics}

Our first description of characteristic subobjects $C$ of an object $A$
describes $C$ as invariant under automorphisms of $A$.   This is an internal
definition, referring only to $A$.   All this takes place in a category
$\mathsf{C}$.  So what we truly seek are functions $A\mapsto F(A)$ whose outputs
are characteristic subobjects of $A$.  In this way given later objects $B$,
these too admit their own characteristic subobjects $F(B)$ that can be compared
with $F(A)$, specifically for every isomorphism $f:A\to B$ we may assign an
appropriate restriction $F(f):F(A)\to F(B)$. This calls for a functor
$F:\mathsf{C}\to \mathsf{C}$.  The second implied concept is the ability to
return to $A$ from $F(A)$, i.e. to produce an actual subobject $C\hookrightarrow
A$ from $F(A)$.  This calls for functions $\epsilon_A:F(A)\to A$, that can be
defined in a manner that is uniform as we vary $A$.  Using the identity functor
$1_{\mathsf{C}}:\mathsf{C}\to \mathsf{C}$, this can be realized as the familiar
diagram of natural transformations $\epsilon:F\Rightarrow 1_{\mathsf{C}}$ between
the endo-functors $F$ and $1_{\mathsf{C}}$:
\begin{center}
    \begin{tikzcd}
        \mathsf{C} 
            \arrow[r, bend right=-45, "F"name=F]
            \arrow[r, bend right=45, "1_{\mathsf{C}}"name=1]
         & \mathsf{C}
         %%% 2-cell
         \arrow[Rightarrow, from=F, to=1,"\epsilon"]
    \end{tikzcd}
    $\qquad \equiv\qquad \forall A.\forall B.\forall f:A\to B.\qquad$
    \begin{tikzcd}
        F(A) \arrow[d,"\epsilon_A"]\arrow[r,"F(f)"] & F(B) \arrow[d,"\epsilon_B"]\\
        A\arrow[r,"f"] & B 
    \end{tikzcd}
\end{center}
An endo-functor $F:\mathsf{C}\to \mathsf{C}$ with a natural transformation 
$\epsilon:F\to 1_{\mathsf{C}}$ is called a \emph{counit}.


Now let us attempt to reverse this generalization to see to what degree it may
miss some of our intended structures.  A few technical matters emerge at this
stage.  Most contexts we consider have an obvious substructures, e.g. subgroups
\& subalgebras.  By appealing to a functorial perspective we do better to
convert our thinking to a categorical definition.  For instance a kernel the
algebraic sense must be assumed to now to be in the categorical sense.  These
usually agree but naunces can emerge in some circumstances.  With modest
assumptions on $\mathsf{C}$ we arrive at the following partial reversal of the
above construction.

\begin{prop}
    If $\mathsf{C}$ has (categorical) images and $\epsilon:F\Rightarrow
    1_{\mathsf{C}}$ is a counit then
\begin{enumerate}
    \item the image $\im\eta_A$ is a subobject of $A$; and
    \item if $f:A\to A$ is an endomorphism then $\im\eta_A=\im\eta_{f(A)}$.
\end{enumerate}
\end{prop}

Witness that this concept matches the stronger concept of a \emph{fully invariant}
subobject because it concerns all endomorphisms rather than solely automorphisms.
Thus for example in the category of groups there is no counit that specifies the 
center because centers are in general not fully invariant.  However, in the 
commutator subgroup can be realized as a counit.  Specifically 
\begin{align}
    F(G) & = [G,G] \\
    F(f:G\to H) & =f|_{[G,G]}:[G,G]\to [H,H] \\
    \eta_G &:[G,G]\hookrightarrow G & \eta_G(x)=x.
\end{align}
A simple option to include all characteristic subgroups we can limit the
category $\mathsf{C}$ to the isomorphism groupoid of $\mathsf{C}$. However a
more natural alternative is to consider a dual concept.  The dual of a counit is
a unit $\eta:1_{\mathsf{C}}\Rightarrow F$.  Here we may again capture a notion
of characteristic by appealing to kernels instead of images.

\begin{prop}
    If $\mathsf{C}$ has (categorial) kernels and $\eta:1_{\mathsf{C}}\Rightarrow
    F$ is a unit then
\begin{enumerate}
    \item the kernel $\ker\eta_A$ is a subobject of $A$; and
    \item if $f:A\to A$ is an endomorphism then $\ker \eta_A=\ker\eta_{f(A)}$.
\end{enumerate}
\end{prop}

Witness that centers (and all marginal subgroups) are (categorical) kernels.

For a categories $\mathsf{C}$ and $\mathsf{D}$, the category of functors
$\mathsf{D}\to \mathsf{C}$ we denote as $\mathsf{C}^{\mathsf{D}}$.  Its objects
are functors and its morphisms are natural transformations.  Now we are interested 
in only those functors which a natural transformation to the identity functor.  
So we are opperating within the \emph{slice} category $\mathsf{C}^{\mathsf{C}}/_1$
for counits, and $1\backslash \mathsf{C}^{\mathsf{C}}$ for units.  We shall give these 
more intuitive names:
\begin{align}
    Char(\mathsf{C}) & = \mathsf{C}^{\mathsf{C}}/_1 & 
    Cochar(\mathsf{C}) & = 1\backslash \mathsf{C}^{\mathsf{C}}.
\end{align}
Our objective is to define elements of these categories and investigate the
subcategories they generate.

\subsection{Operations}

In a group we know the characteristic subgroups form a lattice under joins and
intersections. As we shift to an abstracted point of view that is independent of
the automorphisms and even subsets we need to replace this structure with
operations between the many functors.  This all takes place inside the category
$Char(\mathsf{C})$ and we therefore want a standard categorical product on this
category.  This exists provided that $\mathsf{C}$ has a categorical product.
The diagram in the functor category is thus a pullback along the two counits
$F\Rightarrow 1\Leftarrow G$.  In the usual way such a product is associative.  
If we have a zero in $\mathsf{C}$ then we furthermore have a product identity 
so $Char(\mathsf{C})$ is a monoid.

\subsection{Wiskers and adjoints}

A common and diverse source of units and counits are adjoint functor pairs 
$F\dashv G:\mathsf{C}\to \mathsf{D}$.  That is $1_{\mathsf{C}}\Rightarrow G\circ F$ 
and $F\circ G\Rightarrow 1_{\mathsf{D}}$.  This can be used to give a round-about 
alternative proof that various subgroups are characteristic, but the real value is 
to link with much effort to unused areas.

For instance, free-forgetful adjoint pairs recover the usual verbal and marginal 
subgroups as characteristic subgroups.  We already know this but it may help to follow
such an example to make the idea clear.  Meanwhile another source of adjoint functor pairs 
are induction-restriction functors.  Those are imposed on modules but can also be taken to 
act regularly.

\section{Filters}

In some contexts it may be profitable to impose conditions that the cast of characteristic
functors we devise preserve algebraic structure.  For example, that it commute with 
some operators.  In the category of rings a natural choice is the product, and in groups 
it is the commutator that we consider.


\end{document}